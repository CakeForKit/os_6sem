\chapter{Структура FILE}

\begin{lstlisting}[caption={Структура FILE}, label=lst:FILE]
/* /usr/include/x86_64-linux-gnu/bits/types/FILE.h */
#ifndef __FILE_defined
#define __FILE_defined 1

struct _IO_FILE;

/* The opaque type of streams.  This is the definition used elsewhere.  */
typedef struct _IO_FILE FILE;

#endif


/* /usr/include/x86_64-linux-gnu/bits/types/struct_FILE.h */
struct _IO_FILE
{
	int _flags;		/* High-order word is _IO_MAGIC; rest is flags. */
	
	/* The following pointers correspond to the C++ streambuf protocol. */
	char *_IO_read_ptr;	/* Current read pointer */
	char *_IO_read_end;	/* End of get area. */
	char *_IO_read_base;	/* Start of putback+get area. */
	char *_IO_write_base;	/* Start of put area. */
	char *_IO_write_ptr;	/* Current put pointer. */
	char *_IO_write_end;	/* End of put area. */
	char *_IO_buf_base;	/* Start of reserve area. */
	char *_IO_buf_end;	/* End of reserve area. */
	
	/* The following fields are used to support backing up and undo. */
	char *_IO_save_base; /* Pointer to start of non-current get area. */
	char *_IO_backup_base;  /* Pointer to first valid character of backup area */
	char *_IO_save_end; /* Pointer to end of non-current get area. */
	
	struct _IO_marker *_markers;
	
	struct _IO_FILE *_chain;
	
	int _fileno;
	int _flags2;
	__off_t _old_offset; /* This used to be _offset but it's too small.  */
	
	/* 1+column number of pbase(); 0 is unknown. */
	unsigned short _cur_column;
	signed char _vtable_offset;
	char _shortbuf[1];
	
	_IO_lock_t *_lock;
	#ifdef _IO_USE_OLD_IO_FILE
};
\end{lstlisting}


\chapter{Первая программа}

\section{Код}
\begin{lstlisting}[caption={Программа 1}, label=lst:p1]
#include <stdio.h>
#include <fcntl.h>

int main()
{
	// have kernel open connection to file alphabet.txt
	int fd = open("alphabet.txt",O_RDONLY);
	
	// create two a C I/O buffered streams using the above connection 
	FILE *fs1 = fdopen(fd,"r");
	char buff1[20];
	setvbuf(fs1,buff1,_IOFBF,20);
	
	FILE *fs2 = fdopen(fd,"r");
	char buff2[20];
	setvbuf(fs2,buff2,_IOFBF,20);
	
	// read a char & write it alternatingly from fs1 and fs2
	int flag1 = 1,flag2 = 1;
	while(flag1 == 1 || flag2 == 1) {
		char c;
		flag1 = fscanf(fs1,"%c",&c);
		if (flag1 == 1) {
			fprintf(stdout,"%c",c);
		}
		flag2 = fscanf(fs2,"%c",&c);
		if (flag2 == 1) { 
			fprintf(stdout,"%c",c); 
		}
	}
	return 0;
}
\end{lstlisting}


\section{Результат работы программы}
\FloatBarrier
\imgw{\widthone\textwidth}{img/p1}{Результат работы программы 1}
\FloatBarrier

\section{Анализ полученного результата}
Вызывается системный вызов open, который открывает файл "alphabet.txt" для чтения и возвращает дескриптор в таблице открытых файлов процесса (fd). 

Два раза вызывается функция fdopen, которой передается полученный файловый дескриптор fd. В результате происходит переход от небуферизованного ввода-вывода к буферизованному и возвращаются две структуры FILE (fs1, fs2) библиотеки буферизованного ввода-вывода stdio. Вызывается 2 раза функция setvbuf с флагом \_IOFBF, который определяет тип буферизации -- полный. Данная функция ограничивает размер буфера - 20 байт.

В цикле поочередно вызывается функция fscanf с параметрами fs1 и fs2. В первый буфер записываются 20 символов ("Abcdefghijklmnopqrst"), во второй буфер записываются оставшиеся 6 символов ("uvwxyz").

В результате символы из буферов будут поочередно выведены на экран.

\clearpage
\section{Связь структур файловой подсистемы}
\FloatBarrier
\imgw{\widthone\textwidth}{img/sch1.pdf}{Связь структур файловой подсистемы}
\FloatBarrier



\chapter{Вторая программа, вариант 1}

\section{Однопоточная версия}

\subsection{Код}
\begin{lstlisting}[caption={Программа 2, вариант 1, однопоточной версия}, label=lst:p21]
#include <fcntl.h>
#include <unistd.h>

int main()
{
	char c;  
	// have kernel open two connection to file alphabet.txt  
	int fd1 = open("alphabet.txt",O_RDONLY);
	int fd2 = open("alphabet.txt",O_RDONLY);
	int fl1 = 1,fl2 = 1;
	while(fl1 || fl2) {
		fl1 = read(fd1,&c,1);
		if (fl1 == 1) 
		write(1,&c,1);
		fl2 = read(fd2,&c,1);
		if (fl2 == 1) 
		write(1,&c,1);
	}
	return 0;
}
\end{lstlisting}

\subsection{Результат работы программы}
\FloatBarrier
\imgw{\widthone\textwidth}{img/p21}{Результат работы однопоточной версии программы 2, вариант 1}
\FloatBarrier

\subsection{Анализ полученного результата}

Два раза вызывается системный вызов open, который открывает файл "alphabet.txt" для чтения и возвращает дескрипторы в таблице открытых файлов процесса (fd1 и fd2). Оба дескриптора ссылаются на один и тот же обьект inode, но создаются 2 записи в таблице дескрипторов открытых файлов процесса обьекта files\_struct. Так как обьекты struct file соответствующие дескрипторами fd1 и fd2 разные, они имеют разные поля f\_pos. 

Вызывается системный вызов read с параметром fd1 и из файла считывается симовл 'A', указатель f\_pos обьекта struct file, соответствующий fd1, смещается на 1. Символ выводится на экран. Вызывается системный вызов read с параметром fd2. Указатель f\_pos, соответствующий fd2, равен 0 и из файла считывается симовл 'A', этот указатель смещается на 1. Символ выводится на экран.

В результате каждая буква алфавита будет выведена 2 раза, так как использовалось 2 разных обьекта struct file, каждый со своим значением поля f\_pos.

\section{Многопоточная версия}
\subsection{Код}
\begin{lstlisting}[caption={Программа 2, вариант 1, многопоточная версия}, label=lst:p22]
#include <fcntl.h>
#include <unistd.h>
#include <pthread.h>
#include <stdio.h>
#include <stdlib.h>

void *thread_func(void *arg)
{
	int fd = *(int *)arg;
	char c;
	int fl = 1;
	while(fl) {
		fl = read(fd,&c,1);
		if (fl == 1) 
		write(1,&c,1);
	}
	return NULL;
}

int main()
{
	int fd1 = open("alphabet.txt",O_RDONLY);
	int fd2 = open("alphabet.txt",O_RDONLY);
	pthread_attr_t attr;
	pthread_attr_init(&attr);
	pthread_attr_setdetachstate(&attr, PTHREAD_CREATE_DETACHED);
	pthread_t t;
	if (pthread_create(&t, &attr,thread_func,&fd1)) {
		perror("pthread_create");
		exit(1);
	}
	char c;
	int fl = 1;
	while(fl) {
		fl = read(fd2,&c,1);
		if (fl == 1) 
		write(1,&c,1);
	}
	return 0;
}
\end{lstlisting}

\subsection{Результат работы программы}
\FloatBarrier
\imgw{\widthone\textwidth}{img/p22}{Результат работы многопоточной версии программы 2, вариант 1}
\FloatBarrier

\subsection{Анализ полученного результата}
Два раза вызывается системный вызов open, который открывает файл "alphabet.txt" для чтения и возвращает дескрипторы в таблице открытых файлов процесса (fd1 и fd2). Оба дескриптора ссылаются на один и тот же обьект inode, но создаются 2 записи в таблице дескрипторов открытых файлов процесса обьекта files\_struct, у каждого обьекта files\_struct свое значение f\_pos.

Создается отсоединенный поток, который выполняет то же самое что и основной, за исключением того, что отсоединенный поток работает с дескриптором fd1, а основной с fd2. В каждом потоке вызывается системный вызов read и из соответствующего файла считывается символ, указатель f\_pos обьекта struct file смещается на 1. Символ выводится на экран. Так как на создание потока требуется время, то главный поток успевает прочитать и вывести часть символов. Далее потоки поочерёдно читают и выводят символы, при этом когда главный поток заканчивает чтение, процесс завершается и дополнительный поток не успевает дочитать до конца файла.

\chapter{Вторая программа, вариант 2}
\section{Однопоточная версия}
\subsection{Код}
\begin{lstlisting}[caption={Программа 2, вариант 2, однопоточная версия}, label=lst:p22_one]
#include <fcntl.h>
#include <unistd.h>

int main() 
{
	int fd1 = open("q.txt",O_RDWR);
	int fd2 = open("q.txt",O_RDWR);
	
	for(char c = 'a'; c <= 'z'; c++) {
		if (c%2){
			write(fd1, &c, 1);
		} else {
			write(fd2, &c, 1);
		}
	}
	close(fd1);
	close(fd2);
	return 0;
}
\end{lstlisting}

\subsection{Результат работы программы}
\FloatBarrier
\imgw{\widthone\textwidth}{img/p22_one}{Результат работы однопоточной версии программы 2, вариант 2}
\FloatBarrier

\subsection{Анализ полученного результата}

Два раза вызывается системный вызов open, который открывает файл "q.txt" для чтения и записи и возвращает дескрипторы в таблице открытых файлов процесса (fd1 и fd2). Оба дескриптора ссылаются на один и тот же обьект inode, но создаются 2 записи в таблице дескрипторов открытых файлов процесса обьекта files\_struct, у каждого обьекта files\_struct свое значение f\_pos.

В начале цикла системному вызову write передается fd2, так как 'a' = 97, 97 \% 2 == 1. В начало файла записывается символ 'а', f\_pos обьекта struct file, соответствующего дескриптору fd2, смещается на 1.

Во втором проходе цикла системному вызову write передается fd1. В начало файла записывается символ 'b', так как f\_pos обьекта struct file, соответствующего дескриптору fd1 равен 0. f\_pos смещается на 1.

В результате в файл будут записаны только символы, стоящие на чётных позициях в алфавите.

\section{Многопоточная версия}
\subsection{Код}
\begin{lstlisting}[caption={Программа 2, вариант 2, многопоточная версия}, label=lst:p22_th]
#include <fcntl.h>
#include <unistd.h>
#include <pthread.h>
#include <stdio.h>
#include <stdlib.h>

void *thread_func(void *arg)
{
	int fd = *(int *)arg;
	for(char c = 'a'; c <= 'z'; c++) {
		if (c % 2 == 0) {
			write(fd, &c, 1);
		}
	}
	return NULL;
}

int main() 
{
	int fd1 = open("q.txt",O_RDWR);
	int fd2 = open("q.txt",O_RDWR);
	pthread_attr_t attr;
	pthread_attr_init(&attr);
	pthread_attr_setdetachstate(&attr, PTHREAD_CREATE_DETACHED);
	pthread_t t1;
	if (pthread_create(&t1, &attr,thread_func,&fd1)) {
		perror("pthread_create");
		exit(1);
	}
	for(char c = 'a'; c <= 'z'; c++) {
		if (c%2 == 1) {
			write(fd2, &c, 1);
		}
	}
	close(fd1);
	close(fd2);
	return 0;
}
\end{lstlisting}

\subsection{Результат работы программы}
\FloatBarrier
\imgw{\widthone\textwidth}{img/p22_th}{Результат работы многопоточная версии программы 2, вариант 2}
\FloatBarrier

\subsection{Анализ полученного результата}

Два раза вызывается системный вызов open, который открывает файл "q.txt" для чтения и записи и возвращает дескрипторы в таблице открытых файлов процесса (fd1 и fd2). Оба дескриптора ссылаются на один и тот же обьект inode, но создаются 2 записи в таблице дескрипторов открытых файлов процесса обьекта files\_struct, у каждого обьекта files\_struct свое значение f\_pos.

Создается отсоединенный поток, который выполняет то же самое что и основной, за исключением того, что отсоединенный поток работает с дескриптором fd1, а основной с fd2. Потоки параллельно записывают в один файл и, так как они используют разные индексы открытых файлов, то один поток перезаписывает символы другого.

Таким образом в файл будут записаны символы, стоящие как на чётных так и на нечетных позициях в алфавите.

%\clearpage
\section{Связь структур файловой подсистемы}
\FloatBarrier
\imgw{\widthone\textwidth}{img/sch2.pdf}{Связь структур файловой подсистемы}
\FloatBarrier


\chapter{Третья программа}

\section{Код}
\begin{lstlisting}[caption={Программа 3}, label=lst:p3]
#include <fcntl.h>
#include <unistd.h>
#include <stdio.h>

int main() 
{
	FILE *f1, *f2;
	f1 = fopen("f.txt","w");
	f2 = fopen("f.txt","w");
	for(char c = 'a'; c <= 'z'; c++) {
		if (c%2)
			fprintf(f1, "%c", c);
		else
			fprintf(f2, "%c", c);
	}
	printf("fclose(f1); fclose(f2);\n");
	fclose(f1);
	fclose(f2);
	return 0;
}
\end{lstlisting}

\section{Результат работы программы}
\FloatBarrier
\imgw{\widthone\textwidth}{img/p3}{Результат работы программы 3}
\FloatBarrier

\section{Анализ полученного результата}
Два раза вызывается функция fopen, которая открывает файл "f.txt" и инициализирует 2 структуры \_IO\_FILE (f1 и f2). В цикле в буфер структуры f1 записываются символы, стоящие на чётных позициях в алфавите, а в буфер структуры f2 -- на нечетных. 

Так как используется буферизация, то содержимое, записанное в буфер обьекта \_IO\_FILE будет записано в файл в одном из следующих случаев:
\begin{itemize}[topsep=0pt, partopsep=0pt, itemsep=0pt, parsep=0pt]
	\item буфер заполнен;
	\item вызов функции fflush();
	\item  вызов функции fclose(); 
\end{itemize}

В результате в программе содержимое файла определяется порядком вызова fclose(): если первым вызвать fclose для f1, то при вызове fclose() для f2 содержимое файла будет перезаписано содержимым буфера f2 и наоборот.

Чтобы избежать потери данных, файл следует открывать в режиме добавления - O\_APPEND, в таком случае запись каждый раз будет производиться в реальный конец файла.

\section{Связь структур файловой подсистемы}
\FloatBarrier
\imgw{\widthone\textwidth}{img/sch3.pdf}{Связь структур файловой подсистемы}
\FloatBarrier