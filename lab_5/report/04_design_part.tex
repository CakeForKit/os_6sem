\chapter{Конструкторская часть}

В данном разделе представлены описание сущностей базы данных и их связей, ограничений целостности базы данных, ролевой модели базы данных и функции.


\section{Описание сущностей базы данных и их связей}

На основе формализации задачи в аналитической части были выделены сущности, их атрибуты и связи между ними в виде диаграммы сущность-связь в нотации Чена~\ref{img:img/ER_RU}.

Описание выделенных сущностей представлено в таблицах~\ref{tbl:Collection}-\ref{tbl:Admin}.

\FloatBarrier
\imgw{\widthone\textwidth}{img/ER_RU}{Диаграмма сущность-связь в нотации Чена}
\FloatBarrier



Таблица Collection~\ref{tbl:Collection} содержит информацию о коллекциях на которые разбиты произведения искусств внутри музея.

\begin{longtable}{|
		>{\centering\arraybackslash}m{.33\textwidth - 2\tabcolsep}|
		>{\centering\arraybackslash}m{.34\textwidth - 2\tabcolsep}|
		>{\centering\arraybackslash}m{.33\textwidth - 2\tabcolsep}|
	}
	\caption{Таблица Коллекция (Collection)}\label{tbl:Collection} \\\hline
	Поле & Описание & Тип и ограничение  \\\hline    
	\endfirsthead
	\caption*{Продолжение таблицы~\ref{tbl:Collection} } \\\hline
	Поле & Описание & Тип и ограничение  \\\hline           
	\endhead
	\endfoot
	idCollection & Идентификатор коллекции & Уникальное в таблице целое число \\\hline
	nameCollection & Название коллекции & Строковое поле \\\hline
\end{longtable}

Таблица Author~\ref{tbl:Author} содержит информацию о авторах чьи произведения искусства есть в музее.

\begin{longtable}{|
		>{\centering\arraybackslash}m{.33\textwidth - 2\tabcolsep}|
		>{\centering\arraybackslash}m{.34\textwidth - 2\tabcolsep}|
		>{\centering\arraybackslash}m{.33\textwidth - 2\tabcolsep}|
	}
	\caption{Таблица Автор (Author)}\label{tbl:Author} \\\hline
	Поле & Описание & Тип и ограничение  \\\hline    
	\endfirsthead
	\caption*{Продолжение таблицы~\ref{tbl:Author} } \\\hline
	Поле & Описание & Тип и ограничение  \\\hline           
	\endhead
	\endfoot
	idAuthor & Идентификатор автора & Уникальное в таблице целое число \\\hline
	nameAuthor & Имя автора & Строковое поле \\\hline
	yearBirth & Год рождения автора & Неотрицательное целое число \\\hline
	yearDeath & Год смерти автора & Неотрицательное целое число или NULL \\\hline
\end{longtable}

Таблица Events~\ref{tbl:Events} содержит информацию о мероприятиях, в которых задействованы экспонаты в определенные период времени. Это могут быть выставки или реставрация. 

\begin{longtable}{|
		>{\centering\arraybackslash}m{.33\textwidth - 2\tabcolsep}|
		>{\centering\arraybackslash}m{.34\textwidth - 2\tabcolsep}|
		>{\centering\arraybackslash}m{.33\textwidth - 2\tabcolsep}|
	}
	\caption{Таблица Мероприятия (Events)}\label{tbl:Events} \\\hline
	Поле & Описание & Тип и ограничение  \\\hline    
	\endfirsthead
	\caption*{Продолжение таблицы~\ref{tbl:Events} } \\\hline
	Поле & Описание & Тип и ограничение  \\\hline           
	\endhead
	\endfoot
	idEvent & Идентификатор мероприятия & Уникальное в таблице целое число \\\hline
	title & Название мероприятия & Строковое поле \\\hline
	dateBegin & Дата и время начала мероприятия & Дата и время \\\hline
	dateEnd & Дата и время окончания мероприятия & Дата и время \\\hline
	adress & Адрес проведения мероприятия & Строковое поле \\\hline
	canVisit & Возможность посмотреть экспонат & Логическое поле \\\hline
	creatorID & Идентификатор сотрудника, который зарегистрировал мероприятие & Уникальное в таблице целое число \\\hline
	cntTickets & Количество билетов на мероприятие & Целое неотрицательное число \\\hline
\end{longtable}

Таблица Artwork~\ref{tbl:Artwork} содержит информацию об экспонатах, которые хранятся в музее.

\begin{longtable}{|
		>{\centering\arraybackslash}m{.33\textwidth - 2\tabcolsep}|
		>{\centering\arraybackslash}m{.34\textwidth - 2\tabcolsep}|
		>{\centering\arraybackslash}m{.33\textwidth - 2\tabcolsep}|
	}
	\caption{Таблица Экспонатов (Artwork)}\label{tbl:Artwork} \\\hline
	Поле & Описание & Тип и ограничение  \\\hline    
	\endfirsthead
	\caption*{Продолжение таблицы~\ref{tbl:Artwork} } \\\hline
	Поле & Описание & Тип и ограничение  \\\hline           
	\endhead
	\endfoot
	idArtwork & Инвентарный номер экспоната & Уникальное в таблице целое число \\\hline
	nameArtwork & Название экспоната & Строковое поле \\\hline
	yearCreate & Приблизительный год создания экспоната & Неотрицательное целое число \\\hline
	technic & Техника создания & Строковое поле \\\hline
	size & Размер экспоната & Строковое поле \\\hline
	material & Материал & Строковое поле \\\hline
%	idEvent & Идентификатор мероприятия & Целое число, являющееся идентификатором в таблице мероприятий \\\hline
	idAuthor & Идентификатор автора & Целое число, являющееся идентификатором в таблице авторов \\\hline
	idCollection & Идентификатор коллекции & Целое число, являющееся идентификатором в таблице коллекций \\\hline
\end{longtable}


Таблица Artwork-event~\ref{tbl:ArtworkEvent} содержит информацию о том в каких выставках участвую экспонаты.
\begin{longtable}{|
		>{\centering\arraybackslash}m{.33\textwidth - 2\tabcolsep}|
		>{\centering\arraybackslash}m{.34\textwidth - 2\tabcolsep}|
		>{\centering\arraybackslash}m{.33\textwidth - 2\tabcolsep}|
	}
	\caption{Таблица связи выставки и экспоната~(Artwork-event)}\label{tbl:ArtworkEvent} \\\hline
	Поле & Описание & Тип и ограничение  \\\hline    
	\endfirsthead
	\caption*{Продолжение таблицы~\ref{tbl:ArtworkEvent} } \\\hline
	Поле & Описание & Тип и ограничение  \\\hline           
	\endhead
	\endfoot
	idArtwork & Инвентарный номер экспоната & Уникальное в таблице целое число \\\hline
	idEvent & Идентификатор мероприятия & Уникальное в таблице целое число \\\hline
\end{longtable}

Таблица TicketРurchases~\ref{tbl:TicketРurchases} содержит информацию о купленных билетах.

\begin{longtable}{|
		>{\centering\arraybackslash}m{.33\textwidth - 2\tabcolsep}|
		>{\centering\arraybackslash}m{.34\textwidth - 2\tabcolsep}|
		>{\centering\arraybackslash}m{.33\textwidth - 2\tabcolsep}|
	}
	\caption{Таблица купленых билетов~(TicketРurchases)}\label{tbl:TicketРurchases} \\\hline
	Поле & Описание & Тип и ограничение  \\\hline    
	\endfirsthead
	\caption*{Продолжение таблицы~\ref{tbl:TicketРurchases} } \\\hline
	Поле & Описание & Тип и ограничение  \\\hline           
	\endhead
	\endfoot
	idTicketРurchase & Идентификатор купленного билета & Уникальное в таблице целое число \\\hline
	customerName  & Имя человека, который купил билет & Строковое поле \\\hline
	customerEmail & Почта человека, который купил билет & Строковое поле \\\hline
	eventID & Идентификатор мероприятия & Уникальное в таблице целое число \\\hline
	purchaseDate & Идентификатор мероприятия & Дата и время \\\hline
\end{longtable}

Таблица Employee~\ref{tbl:Employee} содержит информацию о сотрудниках.

\begin{longtable}{|
		>{\centering\arraybackslash}m{.33\textwidth - 2\tabcolsep}|
		>{\centering\arraybackslash}m{.34\textwidth - 2\tabcolsep}|
		>{\centering\arraybackslash}m{.33\textwidth - 2\tabcolsep}|
	}
	\caption{Таблица сотрудников~(Employee)}\label{tbl:Employee} \\\hline
	Поле & Описание & Тип и ограничение  \\\hline    
	\endfirsthead
	\caption*{Продолжение таблицы~\ref{tbl:Employee} } \\\hline
	Поле & Описание & Тип и ограничение  \\\hline           
	\endhead
	\endfoot
	idEmployee & Идентификатор сотрудника & Уникальное в таблице целое число \\\hline
	username  & Имя сотрудника & Строковое поле \\\hline
	login & Логин сотрудника & Строковое поле \\\hline
	hashedPassword & Хеш пароля & Строковое поле \\\hline
	createdAt & Время последнего изменения пароля & Дата и время \\\hline
	valid & Действующий ли сотрудник & Логическое поле \\\hline
	adminID & Идентификатор администратора, который добавил в таблицу & Уникальное в таблице целое число \\\hline
\end{longtable}

Таблица Admin~\ref{tbl:Admin} содержит информацию об администраторах.

\begin{longtable}{|
		>{\centering\arraybackslash}m{.33\textwidth - 2\tabcolsep}|
		>{\centering\arraybackslash}m{.34\textwidth - 2\tabcolsep}|
		>{\centering\arraybackslash}m{.33\textwidth - 2\tabcolsep}|
	}
	\caption{Таблица сотрудников~(Admin)}\label{tbl:Admin} \\\hline
	Поле & Описание & Тип и ограничение  \\\hline    
	\endfirsthead
	\caption*{Продолжение таблицы~\ref{tbl:Admin} } \\\hline
	Поле & Описание & Тип и ограничение  \\\hline           
	\endhead
	\endfoot
	idAdmin & Идентификатор администратора & Уникальное в таблице целое число \\\hline
	username  & Имя администратора & Строковое поле \\\hline
	login & Логин администратора & Строковое поле \\\hline
	hashedPassword & Хеш пароля & Строковое поле \\\hline
	createdAt & Время последнего изменения пароля & Дата и время \\\hline
	valid & Действующий ли сотрудник & Логическое поле \\\hline
\end{longtable}

На диаграмме базы данных~\ref{img:img/ER_tables.pdf} представлены связи между вышеописанными таблицами.

\FloatBarrier
\imgw{\widthone\textwidth}{img/ER_tables.pdf}{Диаграмма проектируемой базы данных}
\FloatBarrier

\section{Описание ограничений целостности базы данных}

В таблице Events не должно быть записей в которых по временным отметках экспонат находится в двух места одновременно.

В таблице Author для каждой записи об авторе должна быть определена хотя бы одна запись о произведении искусства.

В таблице Collection для каждой записи коллекции должна быть определена хотя бы одна запись о произведении искусства.

В таблице TicketРurchases количество записей о купленных билетах на конкретное мероприятие не должно превышать максимальное их количество записанное в поле cntTickerts таблицы Events.

\clearpage

\section{Описание функции}

При добавлении в базу данных записи о связи между таблицами Artwork и Events, информирующей о том что какой-либо экспонат участвует в мероприятии, для сохранения целостности базы данных должна проводиться проверка, что данный экспонат в период проведения мероприятия не участвует в другом мероприятии. Для этого необходима соответствующая функция проверки. Схема алгоритма реализации данной функции представлена на рисунке~\ref{img:img/func}

\FloatBarrier
\imgw{\widthone\textwidth}{img/func}{Схема алгоритма поиска мероприятий, в которых участвует экспонат в заданный период времени}
\FloatBarrier


\section{Описание ролевой модели базы данных}

На основе информации о пользователях проектируемой базы данных, которая была представлена в аналитической разделе на рисунке~\ref{img:img/user-case-3}, были выделены следующий роли и их права доступа:

\begin{itemize}
	\item Пользователь. Имеет права на просмотр таблиц экспонатов, авторов, коллекций, выставок и добавление данных в таблицу купленных билетов;
	\item Сотрудник. Имеет права на просмотр и редактирование таблиц экспонатов, авторов, коллекций, выставок и только просмотр таблицы купленных билетов;
	\item Администратор. Имеет права на просмотр и редактирования таблиц сотрудников, а также просмотр таблицы администраторов; 
\end{itemize}


%Спроектировать функцию, которая принимает 
% на вход индекс экспоната и дату 
% возвращает информацию о выставке, в которой участвует экспонат в это время, если такая есть.



\section*{Вывод}

В данном разделе были описаны сущностей базы данных и их связей, ограничений целостности базы данных, ролевой модели базы данных и функции.

%\section{Математические основы алгоритма обратной трассировки лучей}
%\subsection{Определение пересечения луча с полигоном}
%Так как любую плоскость можно однозначно задать тремя точками, для поиска пересечения луча с полигонами используется алгоритм Моллера~--~Трумбора~\cite{trIntersect}, с помощью которого можно вычислить пересечение луча с треугольным полигоном.
%
%Пусть треугольный полигон определен вершинами $V_0, V_1, V_2$, а луч $R(t)$ с началом в точке $O$ и единичным  вектором направления $D$ определен формулой
%\begin{equation}
%	R(t) = O + tD
%\end{equation}
%
%И если точку $T(u,v)$ на треугольнике $V_0V_1V_2$ выразить через ее барицентрические координаты $(u,v)$, так что ($u \geq 0, v \geq 0, u + v \leq 1$):
%\begin{equation}
%	T(u,v) = (1 - u - v)V_0 + uV_1 + vV_2,
%\end{equation}
%тогда пересечение луча $R(t)$ и треугольника $V_0V_1V_2$ эквивалентно решению уравнения
%$R(t) = T(u, v)$ и однозначно определяются параметрами расстояния $t$ от начала луча до точки пересечения и барицентрическими координатами $(u,v)$. В таком случае получим:
%\begin{equation}\label{eq:eq1}
%	O + tD = (1- u - v)V_0 + uV_1 + vV_2
%\end{equation}
%
%Уравнение~\ref{eq:eq1} может быть представлено в матричном виде:
%\begin{equation}
%	\label{slau}
%	\begin{bmatrix}
%		-D & V_1 - V_0, V_2 - V_0
%	\end{bmatrix}
%	\begin{bmatrix}
%		t\\
%		u\\
%		v
%	\end{bmatrix} = O - V_0 
%\end{equation}
%
%Пусть $E_1 = V_1 - V_0, E_2 = V_2 - V_0$, $T=O - V_0$. Решение уравнения~\ref{eq:eq1} можно получить методом Крамера:
%\begin{equation}\label{eq:eq2}
%	\begin{bmatrix}
%		t\\
%		u\\
%		v
%	\end{bmatrix} = \frac{1}{(D\times E_2) \cdot E_1}
%	\begin{bmatrix}
%		(T\times E_1) \cdot E_2\\
%		(D\times E_2) \cdot T\\
%		(T\times E_1) \cdot D
%	\end{bmatrix}
%\end{equation}
% 
%Если барицентрические координаты точки пересечения, полученной из формулы~\ref{eq:eq2} удовлетворяют условию ($u \geq 0, v \geq 0, u + v \leq 1$), то луч $R(t)$ пересекает треугольный полигон заданный вершинами $V_0, V_1, V_2$. Если определитель $(D\times E_2) \cdot E_1$ равен нулю, то луч лежит в плоскости треугольника $V_0V_1V_2$.
%
%
%\subsection{Определение нормали к плоскости}
%Нормаль к плоскости можно найти как векторное произведение двух векторов, принадлежащих данной плоскости~\cite{rodgersCG}:
%
%Для поиска нормали к полигонам необходимо найти векторное произведение двух векторов, которые лежат на полигоне \cite{rodgers}.
%\begin{equation}
%	N = (V_2 - V_0) \times (V_1 - V_0),
%\end{equation}
%где $V_0, V_1, V_2$ -- вершины полигона.
%
%\subsection{Определение вектора отражения}
%
%Для визуализации отражающих поверхностей в алгоритме обратной трассировки лучей необходим способ определения направления вектора отражения зная луч падения~$l$ и нормаль к поверхности~$n$~\cite{rtOneWeekend}.
%
%\imgw{0.4\textwidth}{reflect}{Расчет направления вектора отражения}
%
%Вектор отражения представляется через разность вектора падения $l$ и вектора нормали $n$, длина которого равняется длине двух проекций вектора $l$ на $n$:
%
%\begin{equation}
%	\label{reflect_ray}
%	r = l - 2 n \cdot \frac{(l, n)}{(n, n)}
%\end{equation}
%
%\clearpage
%\section{Функциональная модель программного обеспечения}
%
%Разрабатываемое программное обеспечение должно генерировать кадр на основе информации о положении в пространстве объектов сцены, интенсивности источника света, вектора направления камеры и множестве полигонов, которыми заданы модели фигур. Для создания изображения программное обеспечение использует алгоритмы испускания луча и обратной трассировки луча, которые заключаются испускании луча для каждого пикселя и вычислении цвета этого пикселя.
%
%На рисунках~\ref{img:1_idef0}-\ref{img:2_idef0} представлена функциональная модель программного обеспечения в нотации IDEF0.
%
%\FloatBarrier
%\imgw{0.9\textwidth}{1_idef0}{Функциональная модель программы в нотации IDEF0, уровень 0}
%\FloatBarrier
%\imgw{0.9\textwidth}{2_idef0}{Функциональная модель программы в нотации IDEF0, уровень 1}
%\FloatBarrier
%
%\clearpage
%\section{Описание алгоритма определения цвета пикселя, методом обратной трассировки лучей}
%
%Для генерации кадра сцены алгоритмом обратной трассировки лучей необходимо для каждого пикселя изображения вычислить его цвет, с использованием алгоритма трассировки луча, который представлен на рисунках~\ref{img:castRay1.pdf}-\ref{img:castRay2.pdf}. Данный алгоритм для каждого луча исходящего от наблюдателя и проходящего через центр пикселя вычисляет цвет этого пикселя на основе положения моделей сцены, их материалов и источника света.
%На рисунке~\ref{img:castRay.pdf} представлена схема алгоритма трассировки луча.
%
%\FloatBarrier
%\imgw{0.7\textwidth}{castRay1.pdf}{Схема алгоритма трассировки луча}
%\FloatBarrier
%\imgw{0.7\textwidth}{castRay2.pdf}{Схема алгоритма трассировки луча}
%\FloatBarrier
%
%\clearpage
%\section{Структура разрабатываемого программного обеспечения}
%
%При разработке программного обеспечения использовался объектно-ориентированный подход и паттерны проектирования, для улучшения декомпозиции задачи и облегчения модификаций кода.
%
%На рисунках~\ref{img:facade}-\ref{img:loadManager} представлена диаграмма классов разрабатываемого программного обеспечения.
%\FloatBarrier
%\imgw{0.8\textwidth}{facade}{Диаграмма классов, которые реализуют доступ графическому интерфейсу пользователя к программному обеспечению}
%\FloatBarrier
%
%\begin{itemize}
%	\item \textbf{FacadeScene} -- класс, реализующий структурный паттерн <<фасад>>. Предоставляет графическому интерфейсу пользователя унифицированный интерфейс к программному обеспечению, реализующему взаимодействие со сценой и отрисовку кадра, с помощью паттерна <<команда>>;
%	\item \textbf{BaseCommand} -- базовый класс, реализующий поведенческий паттерн <<команда>>. Инкапсулирует запрос пользователя на выполнение действия в виде отдельного объекта. В программе также реализовано несколько классов производных от данного, которые отвечают за каждый запрос пользователя по отдельности;
%	\item \textbf{DrawManager} -- класс, реализующий операцию отрисовки сцены;
%	\item \textbf{RayTracing} -- класс, реализующий поведенческий паттерн <<стратегия>>. Определяет реализацию алгоритма обратной трассировки лучей;
%	\item \textbf{Drawer} -- базовый класс, реализующий структурный паттерн <<адаптер>>. Определяет операцию закрашивания пикселей;
%	\item \textbf{QtDrawer} -- производный класс от Drawer. Преобразует интерфейс класса QGraphicViewer библиотеки Qt.
%	\item \textbf{MaterialManager} -- класс, реализующий взаимодействие со множеством материалов;
%	\item \textbf{MaterialSolution} -- класс, определяющий множество возможных материалов, которые могут быть использованы в программе;
%	\item \textbf{TransformManager} -- класс, реализующий операции преобразования объектов сцены;
%	\item \textbf{TransformAction} -- базовый класс, реализующий поведенческий паттерн <<стратегия>>. Определяет алгоритм преобразования объектов сцены;
%	\item \textbf{MoveAction} -- производный класс от TransformAction. Определяет алгоритм переноса объектов сцены;
%	\item \textbf{RotateAction} -- производный класс от TransformAction. Определяет алгоритм вращения объектов сцены;
%\end{itemize}
%
%\FloatBarrier
%\imgw{1\textwidth}{sceneManager}{Диаграмма классов, описывающих обьекты сцены}
%\FloatBarrier
%
%\begin{itemize}
%	\item \textbf{SceneManager} -- класс, реализующий взаимодействие с объектами сцены;
%	\item \textbf{Scene} -- класс сцены. Хранит информацию об объектах сцены;
%	\item \textbf{Camera} -- класс камеры. Хранит информацию о наблюдателе и реализует операцию создания луча;
%	\item \textbf{Ray} -- класс луча. Хранит информацию о начальной точке и векторе направления луча;
%	\item \textbf{Model} -- базовый класс фигуры, которая может быть изображена на сцене;
%	\item \textbf{VolumeModel} -- производный класс от Model, является элементом абстракции структурного паттерна <<мост>>;
%	\item \textbf{VolumeModelImpl} -- базовый класс, являющийся элементом реализации структурного паттерна <<мост>>;
%	\item \textbf{TrianglesModel} -- производный класс от VolumeModelImpl, определяет структуру трехмерной модели, как множество вершин и множество треугольных полигонов;
%	\item \textbf{Triangle} -- класс, описывающий треугольный полигон и операции над ним;
%	\item \textbf{Chessboard} -- класс фигуры шахматной доски, реализующий структурный паттерн <<компоновщик>>. Шахматная доска состоит из трех моделей: модели определяющей деревянную основу доски и двух моделей определяющих множества черных и белых клеток;
%\end{itemize}
%
%\FloatBarrier
%\imgw{1\textwidth}{loadManager}{Диаграмма классов, которые реализуют загрузку моделей фигур на сцену}
%\FloatBarrier
%
%\subsection{Описание классов, используемых в программе}
%\begin{itemize}
%	\item \textbf{LoadManager} -- класс, реализующий загрузку моделей;
%	\item \textbf{ReaderSolution} -- класс, определяющий используемый метод чтения файла;
%	\item \textbf{BaseReaderCreator} -- базовый класс, реализующий порождающий паттерн <<фабричный метод>>. Определяет реализацию класса Reader, которая будет создана;
%	\item \textbf{ReaderCreator} -- производный класс от BaseReaderCreator;
%	\item \textbf{Reader} -- базовый класс, реализующий операцию чтения из файла;
%	\item \textbf{DataChessboardReader} -- производный класс от Reader. Определяет чтение из файла информации о шахматной доске;
%	\item \textbf{VolumeModelReader} -- базовый класс, определяющий операцию чтения информации о трехмерной модели;
%	\item \textbf{TrianglesModelReader} -- производный класс от VolumeModelReader. Определяет чтение из файла информации о трехмерной модели, представленной классом TrianglesModel;
%	\item \textbf{DirectorSolution} -- класс, определяющий используемый метод создания Model;
%	\item \textbf{BaseDirectorCreator} -- базовый класс, реализующий порождающий паттерн <<фабричный метод>>. Определяет реализацию класса Director, которая будет создана;
%	\item \textbf{DirectorCreator} -- производный класс от BaseDirectorCreator;
%	\item \textbf{Director} -- базовый класс, реализующий порождающий паттерн <<строитель>>, элемент <<director>>. Определяет операцию создания фигуры (Model);
%	\item \textbf{ChessboardDirector} -- производный класс от Director. Определяет операцию создания класса шахматной доски (Chessboard);
%	\item \textbf{VolumeModelDirector} -- базовый класс, определяющий операцию создания класса модели (Model);
%	\item \textbf{TrianglesModelDirector} -- производный класс от VolumeModelDirector. Определяет операцию создания класса трехмерной модели, представленной классом TrianglesModel;
%	\item \textbf{VolumeModelBuilder} -- базовый класс, реализующий порождающий паттерн <<строитель>>, элемент <<builder>>. Определяет множество операций необходимых для создания фигуры (Model);
%	\item \textbf{TrianglesModelBuilder} -- производный класс от VolumeModelBuilder. Определяет множество операций необходимых для создания фигуры, представленной классом TrianglesModel;	
%\end{itemize}
%
%\section{Требования к программному обеспечению}
%
%\clearpage
%\section*{Вывод из конструкторской части}
%В данном разделе были представлены математические основы алгоритма обратной трассировки лучей и спроектировано программное обеспечение, которое было описано функциональной моделью, схемой алгоритма обратной трассировки лучей и структурой, представленной в виде диаграммы классов.
%
%\clearpage
