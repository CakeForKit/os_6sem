\chapter{Исследовательская часть}

В данном разделе описано исследование влияния наличия индекса на производительность обработки запросов.

\section{Технические характеристики}
\begin{itemize}
	\item {Гостевая операционная система --- Ubuntu 22.04.5 LTS (WSL 2)}
	\item {Оперативная память (RAM) -- 8,0 ГБ;}
	\item {Процессор -- AMD Ryzen 7 5800H with Radeon Graphics, 3201 МГц, ядер: 8, логических процессоров: 16;}
\end{itemize}

При проведении замеров времени ноутбук был включен в сеть электропитания, и были запущены только встроенное приложение окружения и система замеров времени.

\section{Цель исследования}
Цель исследования -- установление зависимости временных характеристик выполнения запроса определенного вида от наличия/отсутствия индекса на атрибуте eventID отношения artwork\_event при различном количестве записей в этой таблице.

\section{Описание исследования}

В ходе исследования производились следующие действия:

\begin{enumerate}[label={\arabic*)}]
	\item создавался контейнер с чистой базой данных;
	\item применялись миграции, для инициализации схемы базы данных;
	\item рассматривалось 3 случая ограничений на алгоритм соединения таблиц: без ограничений, ограничение на использование только Nested Loop и только Hash Join;
	\item таблица artwork заполнялась 10000 тестовыми записами, а таблицы event -- 1000;
	\item количество записей в отношении artwork\_event, представляющей связь мнигие ко многим таблиц artwork и event, увеличивалось от 1000 до 50000 с шагом 1000;
	\item выполнялась команда ANALYZE для сбора статистики о содержимом таблиц базы данных;
	\item на каждом шаге изменения количества записей в таблице artwork\_event:
	\begin{enumerate}[label={\alph*)}]
		\item с помощью команды EXPLAIN ANALYZE замерялось время выполнения запроса~\ref{lst:sqlQuery}, в качестве идентификатора поиска \texttt{id} в условии запроса генерировался рандомный идентификатор;
		\item для одних входных данных проводилось 10 замеров;
		\item если относительная стандартная ошибка среднего (rse) была >= 5\%, то для этих данных замеры продолжались;
		\item создавался индекс на атрибут eventID отношения artwork\_event;
		\item замеры времени, описанные выше, повторялись;
		\item индекс уничтожался;
		\item в результирующую таблицу заносилось среднее время выполнения запроса для обоих случаев: с индексом и без;
	\end{enumerate}
\end{enumerate}

\begin{lstlisting}[style=sql, caption={SQL-запрос}, label=lst:sqlQuery]
SELECT Artworks.title
FROM Artworks
JOIN Artwork_event
ON Artwork_event.artworkID = Artworks.id
WHERE Artwork_event.eventID = <id>;
\end{lstlisting}

На рисунке~\ref{img:img/01_A0} представлена функциональная модель исследования в нотации IDEF0, характеризующая изменяемые и неизменяемые параметры тестирования.

\FloatBarrier
\imgw{\widthone\textwidth}{img/01_A0}{Функциональная модель исследования в нотации IDEF0}
\FloatBarrier

%В ходе исследования проводились замеры времени выполнения запроса~\ref{lst:sqlQuery} при двух сценариях: с созданием индекса на атрибут eventID отношения artwork\_event и без. В качестве идентификатора поиска $<id>$ в условии запроса использовался рандомный идентификатор, который генерировался для каждого единичного запуска.
%
%Исследование производилось на тестовой базе данных, которая была запущена в отдельном контейнере, и заполнялась искуственно созданными данными. Время обработки запроса определялось с помощью команды EXPLAIN ANALYZE, по принципу: для одних входных данных проводилось 10 замеров и если относительная стандартная ошибка среднего (rse) была >= 5\%, то для этих данных замеры продолжались, в результирующую таблицу заносилось среднее значение.
%
%Замеры времени производились при наличии 10000 записей в таблице artwork и 1000 записей в таблице events. Количество записей в отношении artwork\_event, представляющей связь мнигие ко многим таблиц artwork и events, изменялось от 1000 до 50000 с шагом 1000. На каждом шаге изменения количества записей производились замеры времени выполнения запроса.

\section{Результат исследования}

Результаты замеров времени выполнения запроса~\ref{lst:sqlQuery} в мс без ограничений на алгоритм соединения таблиц приведены в приложении Б в таблице~\hyperref[tbl:timeAllJoin]{Б.1}, с ограничением на использование только алгоритма Hash Join -- в таблице~\hyperref[tbl:timeHashJoin]{Б.2}, с ограничением на использование только алгоритма Nested Loop -- в таблице~\hyperref[tbl:timeNestedLoop]{Б.3}. Алгорим Merge Join в исследовании не рассматриваться, потому что в случае без ограничения, данный алгоритм системой не применялся.
% ~\ref{tbl:queryTime}

Соответствующие таблицам графики представлены на рисунках~\ref{img:img/histogram_alljoin}-\ref{img:img/histogram_nestloop}

\FloatBarrier
\imgw{1.0\textwidth}{img/histogram_alljoin}{Зависимость времени выполнения запроса в мс с индексом и без индекса при изменяемом количестве записей в таблице artwork\_event без ограничений на алгоритм соединения таблиц}
\FloatBarrier
\imgw{1.0\textwidth}{img/histogram_hashjoin}{Зависимость времени выполнения запроса в мс с индексом и без индекса при изменяемом количестве записей в таблице artwork\_event с использованим только алгоритма Hash Join}
\FloatBarrier
\imgw{1.0\textwidth}{img/histogram_nestloop}{Зависимость времени выполнения запроса в мс с индексом и без индекса при изменяемом количестве записей в таблице artwork\_event с использованим только алгоритма Nested Loop}
\FloatBarrier

\section{Анализ результатов}
Из проведённых замеров можно сделать следующие выводы:
\begin{itemize}
	\item {Время обработки запроса в случае отсутствия индекса больше, при любом объеме данных;}
	\item {При увеличении количества записей в таблице artworks\_event более 10000 и без указания ограничения на алгоритм соединения таблиц, алгоритм JOIN автоматически изменяется с Nested Loop на Hash Join. Вследствие чего время обработки резко возрастает;}
	\item {При использовании алгоритма Hash Join, время обработки запроса при наличии индекса не возрастает более 2 мс для рассматриваемого количества записей в таблице artworks\_event.}
\end{itemize}



\section{Вывод из исследовательской части}
 В данном разделе было проведено исследование зависимости времени обработки запроса~\ref{lst:sqlQuery} от наличия/отсутствия индекса на атрибуте eventID отношения artwork\_event при изменяемом количестве записей в таблице: от 1000 до 50000 с шагом 1000 и при различных ограничения на алгоритм соединения таблиц. В результате было определено, что добавление индекса ускоряет обработку рассматриваемых запросов.


\clearpage
