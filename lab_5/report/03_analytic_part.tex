\chapter{Аналитическая часть}

В данном разделе описывается анализ предметной области базы данных музея, формализация задачи и выбор базы данных по модели хранения.

\section{Анализ предметной области}


В данной работе под словом музей понимается учреждение, занимающееся хранением и выставлением на обозрение произведений искусства. Произведением искусства может считаться любой объект, имеющий эстетическую и историческую ценность. Поэтому разные музеи хранят разного рода объекты и на сайтах называют их по разному, например, экспонаты, шедевры, картины, произведения, объекты.
Для создания базы данных музея были рассмотрены аналоги, которые представлены в виде информации на сайте Третьяковской галереи, Пушкинского музея, Эрмитажа и Лувра.

\subsection{Третьяковская галерея}

На сайте Третьяковской галереи~\cite{tretyakovgallery} (изображение страницы сайта представлено на рисунке~\ref{img:img/tretyakov_cite}) можно посмотреть в общем списке информацию об имеющихся в архиве картинах. Для поиска необходимой работы пользователю предоставляются возможности по сортировке и фильтрации картин, по автору, названию, категории и периоду из выпадающего списка. 

Пользователь может получить информацию о каждой имеющейся в галерее картине: размер, материал, техника, автор, название, год создания, инвентарный номер; и о выставках, которые проходят и будут проходить в самой галерее. Также он может купить билет на одну из этих выставок.

% TODO
\FloatBarrier
\imgw{\widthone\textwidth}{img/tretyakov_cite}{Пример страницы сайта Третьяковской галереи с информацией о произведениях искусства}
\FloatBarrier



\subsection{Пушкинский музей}

На сайте Пушкинского музея~\cite{pushkinmuseum} (изображение страницы сайта представлено на рисунке~\ref{img:img/pushkinmuseum_cite}) можно посмотреть в общем списке информацию о хранящихся в нем экспонатах, главными характеристиками которых можно выделить название, коллекцию, автора, страну, год создания, размер, материал, прежнего владельца, музейный номер. По некоторым из данных характеристик возможно осуществление фильтрации (коллекция, автор, страна, прежний владелец) и сортировки (автор, год создания). Также пользователь может получить информацию о наличии произведения искусства в музее в данный момент. И он может посмотреть перечень выставок проходящих в музее по датам и приобрести билеты на них.

% TODO
\FloatBarrier
\imgw{\widthone\textwidth}{img/pushkinmuseum_cite}{Пример страницы сайта Пушкинского музея с информацией о произведениях искусства}
\FloatBarrier

\subsection{Эрмитаж}

На сайте Эрмитажа~\cite{hermitagemuseum} (изображение страницы сайта представлено на рисунке~\ref{img:img/hermitagemuseum_cite}) не представлен общий список имеющихся произведений искусства, экспонаты разбиты на категории внутри, которых возможен поиск по всех доступным на сайте характеристикам объектов искусства, из них главными можно выделить: название, автор, размер, техника, материал, период создания, инвентарный номер. Сортировка экспонатов по каким-либо параметрам не доступна.

На сайте пользователь может получить информацию о выставках, которые проходят или будут проходить, как в Эрмитаже, так и в других музеях, в которых участвуют картины из его коллекции. Также пользователь может приобрести билеты на выставки Эрмитажа.

% TODO
\FloatBarrier
\imgw{\widthone\textwidth}{img/hermitagemuseum_cite}{Пример страницы сайта Эрмитажа с информацией о произведениях искусства}
\FloatBarrier

\subsection{Лувр}

На сайте Лувра~\cite{louvre} (изображение страницы сайта представлено на рисунке~\ref{img:img/louvre_cite}) пользователь может увидеть список имеющихся произведений искусства и их характеристики. Основным можно выделить название, автор, период создания, инвентарный номер, размер, материал, техника, коллекция, страна, прежний владелец, местоположение объекта на момент обновления базы данных.

Пользователь может сортировать экспонаты по названию, автору, дате создания, коллекции, инвентарному номеру и актуальности. Также он может осуществлять фильтрацию по названию, автору, периоду создания, коллекции и месту нахождения.

На сайте можно увидеть выставки, проходящие в данный момент в музее и купить на них билеты. Информации о предстоящих выставках нет.

% TODO
\FloatBarrier
\imgw{\widthone\textwidth}{img/louvre_cite}{Пример страницы сайта Лувра с информацией о произведениях искусства}
\FloatBarrier
%\textit{Пример сайта Лувра, информация о произведениях искусства}

\subsection{Сравнение аналогов}
Для сравнения описанных выше аналогов хранилищ произведений искусства были выбраны критерии:
\begin{enumerate}[label={\arabic*)}]
	\item характеристики произведений искусства представленные на сайте;
	\item параметры сортировки экспонатов;	
	\item параметры фильтрации экспонатов;	
	\item наличие информации о местоположении картин;	
	\item наличие информации о предстоящих выставках;	
%	\item наличие выпадающего списка в фильтрах;	
%	\item возможность увидеть полную информацию о произведении искусства;
	\item наличие информации о текущих выставках;
	\item возможность приобретения билетов на выставки;
%	\item возможность спонсорства (программа “Друг музея”);
\end{enumerate}

\begin{longtable}{|
		>{\centering\arraybackslash}m{.2\textwidth - 2\tabcolsep}|
		>{\centering\arraybackslash}m{.2\textwidth - 2\tabcolsep}|
		>{\centering\arraybackslash}m{.2\textwidth - 2\tabcolsep}|
		>{\centering\arraybackslash}m{.2\textwidth - 2\tabcolsep}|
		>{\centering\arraybackslash}m{.2\textwidth - 2\tabcolsep}|
	}
	\caption{Сравнение аналогов}\label{tbl:cmpAnalogues} \\\hline
	 Критерии сравнения & Третьяковская галерея & Эрмитаж & Пушкинский музей & Лувр \\\hline    
	 \endfirsthead
	 \caption*{Продолжение таблицы~\ref{tbl:cmpAnalogues} } \\\hline
	 Критерии сравнения & Третьяковская галерея & Эрмитаж & Пушкинский музей & Лувр \\\hline           
	\endhead
	\endfoot
	
	1 	& Название, автор, период создания, инвентарный номер, размер, материал, техника  
		& Название, автор, период создания, инвентарный номер, размер, материал, техника  
		& Название, автор, период создания, инвентарный номер, размер, материал, коллекция, страна, прежний владелец 
		& Название, автор, период создания, инвентарный номер, размер, материал, техника, коллекция, страна, прежний владелец, местоположение \\\hline
		
	2 	& Название, автор, период 
		& Нет возможности сортировки 
		& Автор, период создания
		& Название, автор, период создания, коллекция, актуальность, инвентарный номер \\\hline

	3 	& Название, автор, категория, период, тип 
		& Название, автор, период создания, инвентарный номер, размер, материал, техника  
		& Название, автор, коллекция, страна, наличие изображения, прежний владелец
		& Название, автор, период создания, коллекция, местонахождения \\\hline
	4 & Нет & Нет & Только о присутствии/отсутствии в залах музея & Есть \\\hline
	5 & - & + & + & -\\\hline
	6 & + & + & + & +\\\hline
	7 & + & + & + & +\\\hline
%	5 & + & - & + & +\\\hline наличие выпадающего списка в фильтрах;	
%	6 & + & + & + & +\\\hline возможность увидеть полную информацию о произведении искусства;
%	9 & + & + & + & +\\\hline возможность спонсорства (программа “Друг музея”);
	
\end{longtable}

%Сравнение существующих решений представлено в таблице~\ref{tbl:cmpAnalogues}. У каждого рассмотренного сайта музея,есть недостатки по некоторым из описанных выше критериям, что может вызвать неудобство у пользователей желающих не просто посетить музей, а увидеть при этом конкретное произведение искусства в живую.

\clearpage
  
\section{Формализация задачи}
База данных музея содержит в себе информацию о произведения искусства и мероприятиях, в которых они участвуют. Каждый экспонат описывается параметрами: название, автор, период создания, коллекция, размер, материал и техника. Для автора определяется его имя и время жизни. Мероприятия описываются названием, временем, местом проведения, статусом, которым обладают участвующие в нем экспонаты (доступен/недоступен для просмотра) и количеством билетов. Также для организации возможности покупки билетов база данных должна содержать реестр купленных билетов.

База данных предназначена для сотрудников музея, администратора и пользователей. 
% авторизованных и неавторизованных гостей музея. 
Пользователи будут использовать разрабатываемое приложение для поиска экспонатов хранящихся в музее, получения информации о них и покупки билетов. Сотрудники будут иметь возможность добавлять, удалять, редактировать экспонаты и мероприятия, в которых они участвуют. Администратор будет отвечать за данные о сотрудниках. 

Информация о пользователях базы данных и их возможностях представлена на диаграмме прецедентов~\ref{img:img/user-case-3}.
%Авторизованные пользователи могут оформлять подписку на рассылку о новых выставках, в которых участвуют экспонаты.

\FloatBarrier
\imgw{\widthone\textwidth}{img/user-case-3}{Пользователи базы данных}
\FloatBarrier
%\textit{Рисунок -- Пользователи базы данных}

\clearpage

\section{Выбор базы данных по модели хранения}

По модели хранения данных БД разделяют на 3 типа~\cite{introDBsys}:

\begin{itemize}
	\item \textbf{Дореляционные базы данных}. Данные формируются в виде структур, наиболее известными примерами которых являются инвертированные списки, деревья и графы. На основе этого определяется способ доступа к данным и работа с ними, в большинстве случаем используются указатели.
%	представляют из себя ранние системы управления данными, включающие инвертированные списки (данные в файлах с индексами), иерархические (организация в виде дерева) и сетевые (представление в виде графа) модели. Они менее гибкие, сложнее в управлении и не поддерживали строгих ограничений целостности, в отличие от реляционных баз данных;
%	
	\item \textbf{Реляционные базы данных}. Данные формируются в виде взаимосвязанных таблиц (отношений), где каждая строка представляет запись, а столбцы -- атрибуты. Работа с данным возможна двумя способами: с использованием реляционной алгебры и реляционных исчислений. Накладываются четкие ограничения на формат представления данных в виде записей с фиксированным количеством атрибутов.
%	организуют данные в виде таблиц (отношений), где каждая строка представляет запись, а столбцы — атрибуты. Такая модель состоит из трех основных компонент: структурной (описывает из каких объектов строится реляционная модель), целостной (определяет 2 базовых требования целостности -- целостность сущности и ссылочную целостность) и манипуляционной (описывает 2 эквивалентных способа манипулирования реляционными данными -- реляционную алгебру и реляционное исчисление);
	\item \textbf{Постреляционные базы данных}. Данные представляются в виде объектов (таблиц), которые могут быть вложенными и содержать переменное количество полей. За счет этого возможно построение зависимостей одних данных от других и создание иерархии.
	
%	преодолевают ограничения реляционной модели. Они поддерживают гибкие структуры данных (документы, графы, ключ-значение), обеспечивают высокую производительность, масштабируемость и отказоустойчивость, подходят для больших объемов данных, неструктурированной информации и специализированных задач.;
\end{itemize}

На основе формализации поставленной задачи и в отсутствии необходимости поддержки гибких иерархичных структур данных была выбрана реляционная модель.

%для реализации зависимостей выделенных элементов друг от друга и так как нет необходимости в поддержке гибких структуры данных, была выбрана реляционная модель.


\section*{Вывод}
В данном разделе были представлены: формализация задачи, анализ предметной области базы данных музея в виде информации на сайте Третьяковской галереи, Пушкинского музея, Эрмитажа и Лувра. Также была выбрана реляционная модель базы данных.


%
%формализация поставленной задаче, анализ существующих решений базы данных музея в виде информации на сайте Третьяковской галереи, Пушкинского музея, Эрмитажа и Лувра. Также была выбрана реляционная модель базы данных.

%\section{Формализация задачи}
%
%\subsection{Функциональные требования к разрабатываемому программному обеспечению}
%
%Программное обеспечение должно предоставлять возможность создавать кадр с изображением сцены, состоящей из моделей шахматных фигур и шахматной доски. Модели шахматных фигур могут находиться только в центре клеток на поверхности доски, должны вращаться вокруг своей оси и иметь возможность поменять позицию. Материал поверхности шахматной доски может быть матовым или глянцевым. Источник света на сцене должен быть один и задаваться внутри программы в фиксированной точке пространства вне поля зрения пользователя.
%
%На рисунке~\ref{img:0_idef0} представлена функциональная модель программы в нотации IDEF0, характеризующая требования к программному обеспечению.
%\FloatBarrier
%\imgw{0.9\textwidth}{0_idef0}{Функциональная модель программы в нотации IDEF0}
%\FloatBarrier
%
%\subsection{Формализация объектов сцены}
%Сцена состоит из следующих объектов:
%\begin{itemize}
%	\item шахматная доска 8x8 клеток -- снование шахматной доски цвета дерева, клетки -- черного и белого цвета;
%	\item набор шахматных фигуры двух цветов: пешка, ладья, конь, слон, ферзь, король;
%	\item точечный источник света -- задается положением в пространстве и интенсивностью излучения;
%	\item камера -- задается положением в пространстве и вектором взгляда.
%\end{itemize}
%
%Форма и размер шахматных фигур и шахматной доски соответствует стандарту шахматного оборудования и игровых площадок, предназначенных для проведения турниров ФИДЕ~\cite{FIDE2015}.
%
%\clearpage
%\section{Анализ способов представления поверхностей трехмерных моделей}
%Поверхность трехмерной модели можно задать несколькими способами~\cite{deymin}:
%\begin{itemize}
%	\item \textbf{Полигональная сетка.} В данном случае поверхность представляется как совокупность связанных между собой плоских многоугольников. Большинство объектов, не имеющих изгибов, например, таких как шахматная доска, можно точно описать полигональной сеткой. Этот способ также применяется для представления объектов, ограниченных криволинейными
%	поверхностями, однако в таком случае объект будет описан достаточно приблизительно.
%	
%	\item \textbf{Аналитический способ.} Поверхность, заданная таким способом описывается функцией зависимости координат от некоторого параметра. Главным достоинством данного метода является высокая точность описания поверхности, которая нужна в большинстве вычислительных программ, однако из-за необходимости проведения большого количества математических вычислений при визуализации данных поверхностей, время их отрисовки будет значительно больше времени отрисовки поверхностей заданных множеством полигонов.
%\end{itemize}
%
%\subsection*{Вывод}
%Для решения задачи визуализации шахматной доски и шахматных фигур нет большой необходимости в точности представления поверхностей. Именно поэтому для уменьшения времени отрисовки сцены был выбран способ задания поверхностей моделей полигональной сеткой.
%
%\clearpage
%\section{Анализ алгоритмов удаления невидимых линий и поверхностей}
%Для создания реалистичного изображения необходимо учитывать такие факторы как невидимые линии и поверхности, тени, освещение, свойства материалов объекта (способность отражать и преломлять свет).
%
%\subsection{Алгоритм Робертса}
%
%Алгоритм Робертса решает задачу удаления невидимых линий и работает в объектном пространстве исключительно с выпуклыми телами, если тело является не выпуклым, то его нужно разбить на выпуклые составляющие~\cite{rodgersCG}.
%
%\textbf{Этапы алгоритма:}
%\begin{enumerate}[label=\arabic*)]
%	\item \textbf{Подготовка исходных данных.} В данном алгоритме выпуклое многогранное тело представляется набором пересекающихся плоскостей. Формируется матрица тела~$V$, где каждый столбец содержит коэффициенты уравнения плоскости грани~$ax + by + cz + d = 0$:
%	\begin{equation}
%		V = \begin{pmatrix}
%			a_{1} & a_{2} & \ldots & a_{n}\\
%			b_{1} & b_{2} & \ldots & b_{n}\\
%			c_{1} & c_{2} & \ldots & c_{n}\\
%			d_{1} & d_{2} & \ldots & d_{n}
%		\end{pmatrix}.
%	\end{equation}
%	\item \textbf{Удаление ребер, экранируемых самим телом.} Используется вектор взгляда~$E = (0, 0, -1, 0)$ для определения невидимых граней. При умножении вектора $E$ на матрицу тела $V$ отрицательные компоненты полученного вектора будут соответствовать невидимым граням;
%	
%	\item \textbf{Удаление невидимых ребер, экранируемых другими телами.} Для определения невидимых точек ребра строится луч, соединяющий точку наблюдения с точкой на ребре. Точка невидима, если луч на своём пути встречает в качестве преграды тело, т.е. проходит через него;
%\end{enumerate}
%
%\textbf{Преимущества алгоритма Робертса:}
%\begin{itemize}
%	\item Высокая точность благодаря работе в объектном пространстве;
%\end{itemize}
%
%\textbf{Недостатки алгоритма Робертса:}
%\begin{itemize}
%	\item Не работает с невыпуклыми телами;
%	\item Невозможность визуализации отражающих поверхностей;
%	\item Квадратичная сложность по числу объектов;
%\end{itemize}
%
%
%\subsection{Алгоритм, использующий z-буфер}
%Алгоритм Z--буфера работает в пространстве изображений и использует два буфера: буфер кадра для хранения цвета каждого
%пикселя и Z--буфер для хранения глубины каждого пикселя \cite{shikinStCG}.
%
%\textbf{Этапы алгоритма:}
%\begin{enumerate}[label=\arabic*)]
%	\item Инициализация Z--буфера минимально возможными значениями и буфера кадра значениями пикселя, описывающими фон;
%	\item Преобразование каждой проекции грани многоугольников в растровую форму;
%	\item Вычисление для каждого пикселя с координатами $(x, y)$, его глубины $Z(x, y)$.
%	\item Сравнение глубины $Z(x, y)$ новых пикселей с текущими в Z--буфере и обновление буфера кадра при необходимости;
%\end{enumerate}
%
%\textbf{Преимущества алгоритма Z--буфера:}
%\begin{itemize}
%	\item Время работы алгоритма не зависит от разрешения объекта сцены (числа граней поверхности многогранника)
%	\item Линейная сложность по числу точек растра и усредненного числа граней поверхности многогранника;
%	\item Простота алгоритма и используемого в нем набора операций;
%\end{itemize}
%
%\textbf{Недостатки алгоритма Z--буфера:}
%\begin{itemize}
%	\item Большой объём требуемой памяти;
%	\item Невозможность визуализации отражающих поверхностей;
%\end{itemize}
%
%\subsection{Алгоритм обратной трассировки лучей}
%Алгоритм обратной трассировки лучей заключается в отслеживании траектории лучей, которые исходят из точки наблюдателя и проходят через центр пикселя растра в направлении сцены. \cite{shikinDinamica}.
%
%\textbf{Этапы алгоритма:}
%\begin{enumerate}[label=\arabic*)]
%	\item Преобразование сцены в пространство изображения, т. е. область видимости наблюдателя разбивается на пиксели.
%	\item Испускание лучей от наблюдателя через пиксели растра к сцене.
%	\item Определение ближайшего пересечения лучей с объектами сцены.
%	\item Определение находится ли точка пересечения в тени путем испускания луча из этой точки в направлении источника света. И если луч пересекает какие-либо объекты сцены, то точка находится в тени.
%	\item Рекурсивное отражение и/или преломление лучей при наличии отражающих или прозрачных материалов.
%	\item Учёт теней путём проверки видимости световых источников из точки пересечения.
%\end{enumerate}
%
%\textbf{Преимущества алгоритма обратной трассировки лучей:}
%\begin{itemize}
%	\item Высокая реалистичность синтезируемого изображения.
%	\item Учет теней и эффектов отражения и преломления.
%\end{itemize}
%
%\textbf{Недостатки алгоритма обратной трассировки лучей:}
%\begin{itemize}
%	\item Увеличенное время выполнения из-за рекурсивных вычислений.
%\end{itemize}
%
%\subsection*{Вывод}
%В качестве алгоритма удаления невидимых линий и поверхностей был выбран алгоритм обратной трассировки лучей, так как с его помощью возможно реализовать эффект отражения, необходимый для решения поставленной задачи. А также данный алгоритм не требует дополнительной реализации алгоритмов закраски и построения теней.
%
%\clearpage
%\section{Анализ модели освещения}
%Модель освещения определяет цвет поверхности объекта, отображаемого на экране и бывает двух видов: локальная и глобальная~\cite{rodgersCG}. 
%В локальной модели освещения учитывается только свет падающий от источников и ориентация поверхности в пространстве. В глобальной модели освещения дополнительно еще учитывается свет, отражённый от других поверхностей и/или пропущенный через них.
%
%\subsection{Локальная модель освещения}
%Локальная модель освещения состоит из трех компонент~\cite{Phong1975}:
%
%\begin{enumerate}[label=\arabic*)]
%	\item \textbf{Фоновое освещение.} Данная составляющая позволяет учитывать свет постоянной яркости, созданный многочисленными отражениями от различных поверхностей. Такой свет практически всегда присутствует в реальной обстановке. Интенсивность  рассеянного света можно рассчитать по формуле~(\ref{eq:amb})
%	\begin{equation}\label{eq:amb}
%		I_{a} = k_{a} \cdot i_{a}
%	\end{equation}
%	\noindent где 
%	$I_{a}$ --- интенсивность рассеянного света, 
%	$k_{a}$ --- коэффициент фонового освещения, 
%	$i_{a}$ --- интенсивность источника рассеянного света. 
%	
%	\item \textbf{Диффузное отражение.} Идеальное диффузное отражение описывается законом Ламберта, согласно которому падающий свет рассеивается во все стороны с одинаковой интенсивностью. Интенсивность диффузного отражения света можно рассчитать по формуле~(\ref{eq:diff})
%	
%	\begin{equation}\label{eq:diff}
%		I_{d} = k_{d} \cdot I_{l} \cdot \frac{\cos(\overrightarrow{L}, \overrightarrow{N})}{r + k}
%	\end{equation}
%	\noindent где 
%	$I_{d}$ --- интенсивность диффузного отражения света, 
%	$k_{d}$ --- коэффициент диффузного отражения, 
%	$I_{l}$ --- интенсивность точечного источника света,
%	$\overrightarrow{L}$ --- вектор направленный на источник света, 
%	$\overrightarrow{N}$ --- вектор нормали к поверхности,
%	$r$ --- расстояние от центра проекции до поверхности
%	$k$ --- произвольная постоянная.
%	
%	\item \textbf{Зеркальное отражение} Направленное отражение которому на блестящих объектах образовываются блики. Наблюдатель видит зеркально отраженный свет только в том случае, когда угол отражения от идеальной отражающей поверхности равен углу падения. Интенсивность зеркального отражения света можно рассчитать по формуле~(\ref{eq:specular})
%	
%	\begin{equation}\label{eq:specular}
%		I_{s} = k_{s} \cdot I_{l} \cdot \frac{\cos^n(\overrightarrow{R}, \overrightarrow{S})}{r + k},
%	\end{equation}
%	
%	\noindent где 
%	$I_{s}$ --- интенсивность зеркального отражения света,
%	$k_{s}$ --- коэффициент зеркального отражения,
%	$I_{l}$ --- интенсивность точечного источника света,
%	$\overrightarrow{R}$ --- вектор отраженного луча, 
%	$\overrightarrow{S}$ --- вектор направленный на наблюдателя,
%	$n$ ---  степень, аппроксимирующая пространственное распределение зеркально отраженного света,
%	$r$ --- расстояние от центра проекции до поверхности
%	$k$ --- произвольная постоянная.
%	
%	
%\end{enumerate}
%
%\begin{equation}\label{eq:all}
%	I = k_{a} \cdot I_{a} + k_{d} \cdot I_{l} \cdot \frac{\cos(\overrightarrow{L}, \overrightarrow{N})}{r + k} + k_{s} \cdot I_{l} \cdot \frac{\cos^n(\overrightarrow{R}, \overrightarrow{S})}{r + k} + k_{s} \cdot I_{r} + k_{t} \cdot I_{t}
%\end{equation}
%
%\subsection{Глобальная модель освещения}
%Глобальная модель освещения дополняет локальную модель и позволяет учитывать положение объектов сцены относительно друг друга, благодаря чему появляется возможность визуализировать эффекты отражения света от других объектов и пропускания света сквозь прозрачные объекты.
%
%Глобальная модель освещения складывается из непосредственной освещенности точки источником света, которая рассчитывается по локальной модели освещения, и вторичной освещенности, которая в свою очередь состоит из интенсивности света отраженного и преломленного луча~\cite{shikin2001}. Интенсивность света в точке по глобальной модели освещения рассчитывается  формулой~(\ref{eq:global}):
%
%\begin{equation}\label{eq:global}
%	I = I_{a} + I_{d} + I_{s} + k_{s} \cdot I_{r} + k_{t} \cdot I_{t},
%\end{equation}
%\noindent где 
%	$I_{a}$ --- интенсивность рассеянного света~(\ref{eq:amb}), 
%	$I_{d}$ --- интенсивность диффузного отражения света~(\ref{eq:diff}), 
%	$I_{s}$ --- интенсивность зеркального отражения света~(\ref{eq:specular}),
%	$k_{s}$ --- коэффициент зеркального отражения,
%	$k_{t}$ --- коэффициент пропускания,
%	$I_{r}$ --- интенсивности света отраженного луча,
%	$I_{t}$ --- интенсивности света преломленного луча.
%\subsection*{Вывод}
%
%
%
%\clearpage
%\section*{Вывод из аналитической части}
%В данном разделе была формализована поставленная задача, были рассмотрены способы представления поверхностей трехмерных моделей и методы создания трёхмерного реалистичного изображения. В результате был выбран метод представления трехмерных поверхностей полигональной сеткой и для визуализации трехмерной сцены был выбран алгоритм обратной трассировки лучей, который включает в себя глобальную модель освещения.
%
%\clearpage
